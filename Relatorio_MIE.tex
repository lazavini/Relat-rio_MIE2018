%%%%%%%%%%%%%%%%%%%%%%%%%%% asme2ej.tex %%%%%%%%%%%%%%%%%%%%%%%%%%%%%%%
% Template for producing ASME-format journal articles using LaTeX    %
% Written by   Harry H. Cheng, Professor and Director                %
%              Integration Engineering Laboratory                    %
%              Department of Mechanical and Aeronautical Engineering %
%              University of California                              %
%              Davis, CA 95616                                       %
%              Tel: (530) 752-5020 (office)                          %
%                   (530) 752-1028 (lab)                             %
%              Fax: (530) 752-4158                                   %
%              Email: hhcheng@ucdavis.edu                            %
%              WWW:   http://iel.ucdavis.edu/people/cheng.html       %
%              May 7, 1994                                           %
% Modified: February 16, 2001 by Harry H. Cheng                      %
% Modified: January  01, 2003 by Geoffrey R. Shiflett                %
% Use at your own risk, send complaints to /dev/null                 %
%%%%%%%%%%%%%%%%%%%%%%%%%%%%%%%%%%%%%%%%%%%%%%%%%%%%%%%%%%%%%%%%%%%%%%

%%% use twocolumn and 10pt options with the asme2ej format
\documentclass[twocolumn,10pt]{asme2ej}
\usepackage[utf8x]{inputenc}
\usepackage{epsfig} %% for loading postscript figures

%% The class has several options
%  onecolumn/twocolumn - format for one or two columns per page
%  10pt/11pt/12pt - use 10, 11, or 12 point font
%  oneside/twoside - format for oneside/twosided printing
%  final/draft - format for final/draft copy
%  cleanfoot - take out copyright info in footer leave page number
%  cleanhead - take out the conference banner on the title page
%  titlepage/notitlepage - put in titlepage or leave out titlepage
%  
%% The default is oneside, onecolumn, 10pt, final


\title{Interfaces Naturais em Mundos Híbridos Credíveis para Educação}

%%% first author
\author{Lázaro V. O. Lima
    \affiliation{
	Professor, Doutoramento em Engenharia Informática\\
	Métodos de Investigação na Engenharia\\
	Universidade do Minho\\
	Davis, California 95616\\
    Email: lazavini@gmailcom.edu
    }	
}


\begin{document}

\maketitle    

%%%%%%%%%%%%%%%%%%%%%%%%%%%%%%%%%%%%%%%%%%%%%%%%%%%%%%%%%%%%%%%%%%%%%%
\begin{abstract}
{\it
This article show the overviem of 
Keywords: Augmented Reality, }
\end{abstract}


%%%%%%%%%%%%%%%%%%%%%%%%%%%%%%%%%%%%%%%%%%%%%%%%%%%%%%%%%%%%%%%%%%%%%%

\section{Introdução }

\subsection{Motivação}
%%%%%%%%%%%%%%%%%%%%%%%%%%%%%%%%%%%%%%%%%%%%%%%%%%%%%%%%%%%%%%%%%%%%%%
\subsection{Objetivos}

\subsection{Hipótese de Pesquisa}



The heading is boldface with upper and lower case letters. 
If the heading should run into more than one line, the run-over is not left-flushed.

%%%%%%%%%%%%%%%%%%%%%%%%%%%%%%%%%%%%%%%%%%%%%%%%%%%%%%%%%%%%%%%%%%%%%%
\section{Estado da Arte}
\subsection{Realidade Aumentada e os Mundos Híbridos}
\subsection{Artigo 1}
\subsubsection{Introdução artigo 1}
\subsubsection{Metodologia}
\subsubsection{}


%%%%%%%%%%%%%%%%%%%%%%%%%%%%%%%%%%%%%%%%%%%%%%%%%%%%%%%%%%%%%%%%%%%%%%
\section{Proposta}
\section{Plano de Trabalho}


%%%%%%%%%%%%%%%%%%%%%%%%%%%%%%%%%%%%%%%%%%%%%%%%%%%%%%%%%%%%%%%%%%%%%%
\section{Footnotes\protect\footnotemark}
\footnotetext{Examine the input file, asme2ej.tex, to see how a footnote is given in a head.}

Footnotes are referenced with superscript numerals and are numbered consecutively from 1 to the end of the paper\footnote{Avoid footnotes if at all possible.}. Footnotes should appear at the bottom of the column in which they are referenced.


%%%%%%%%%%%%%%%%%%%%%%%%%%%%%%%%%%%%%%%%%%%%%%%%%%%%%%%%%%%%%%%%%%%%%%
\section{Mathematics}

\section{Figures}
\label{sect_figure}

All figures should be positioned at the top of the page where possible.  All figures should be numbered consecutively and centered under the figure as shown in Fig.~\ref{figure_ASME}. All text within the figure should be no smaller than 7~pt. There should be a minimum two line spaces between figures and text. The number of a referenced figure or table in the text should be preceded by Fig.\ or Tab.\ respectively unless the reference starts a sentence in which case Fig.\ or Tab.\ should be expanded to Figure or Table.


%%%%%%%%%%%%%%%%%%%%%%%%%%%%%%%%%%%%%%%%%%%%%%%%%%%%%%%%%%%%%%%%%%%%%%
%%%%%%%%%%%%%%%% begin figure %%%%%%%%%%%%%%%%%%%
\begin{figure}[t]
\begin{center}
\setlength{\unitlength}{0.012500in}%
\begin{picture}(115,35)(255,545)
\thicklines
\put(255,545){\framebox(115,35){}}
\put(275,560){Beautiful Figure}
\end{picture}
\end{center}
\caption{The caption of a single sentence does not have period at the end}
\label{figure_ASME} 
\end{figure}
%%%%%%%%%%%%%%%% end figure %%%%%%%%%%%%%%%%%%% 
%%%%%%%%%%%%%%%%%%%%%%%%%%%%%%%%%%%%%%%%%%%%%%%%%%%%%%%%%%%%%%%%%%%%%%



%%% the maximum width in double column is 6.85in
\begin{figure*} 
\centerline{\psfig{figure=figure/FMANU_MD_04_1274_13.ps,width=6.85in}}
\caption{A figure expanded to double column width the text from Figure~\ref{fig_example3.ps}}
\label{fig_example4.ps}
\end{figure*}
%%%%%%%%%%%%%%%% end figure %%%%%%%%%%%%%%%%%%%

%%%%%%%%%%%%%%%%%%%%%%%%%%%%%%%%%%%%%%%%%%%%%%%%%%%%%%%%%%%%%%%%%%%%%%
\section{Citing References}

%%%%%
%%%%%%%%%%%%%%%%%%%%%%%%%%%%%%%%%%%%%%%%%%%%%%%%%%%%%%%%%%%%%%%%%%%%%%
\section{Conclusões}



%%%%%%%%%%%%%%%%%%%%%%%%%%%%%%%%%%%%%%%%%%%%%%%%%%%%%%%%%%%%%%%%%%%%%%
\section{Discussions}
This template is not yet ASME journal paper format compliant at this point.
More specifically, the following features are not ASME format compliant.
\begin{enumerate}
\item
The format for the title, author, and abstract in the cover page.
\item
The font for title should be 24 pt Helvetica bold.
\end{enumerate}

\noindent
If you can help to fix these problems, please send us an updated template.
If you know there is any other non-compliant item, please let us know.
We will add it to the above list.
With your help, we shall make this template 
compliant to the ASME journal paper format.


%%%%%%%%%%%%%%%%%%%%%%%%%%%%%%%%%%%%%%%%%%%%%%%%%%%%%%%%%%%%%%%%%%%%%%
\begin{acknowledgment}
ASME Technical Publications provided the format specifications for the Journal of Mechanical Design, though they are not easy to reproduce.  It is their commitment to ensuring quality figures in every issue of JMD that motivates this effort to have authors review the presentation of their figures.  

Thanks go to D. E. Knuth and L. Lamport for developing the wonderful word processing software packages \TeX\ and \LaTeX. We would like to thank Ken Sprott, Kirk van Katwyk, and Matt Campbell for fixing bugs in the ASME style file \verb+asme2ej.cls+, and Geoff Shiflett for creating 
ASME bibliography stype file \verb+asmems4.bst+.
\end{acknowledgment}

%%%%%%%%%%%%%%%%%%%%%%%%%%%%%%%%%%%%%%%%%%%%%%%%%%%%%%%%%%%%%%%%%%%%%%
% The bibliography is stored in an external database file
% in the BibTeX format (file_name.bib).  The bibliography is
% created by the following command and it will appear in this
% position in the document. You may, of course, create your
% own bibliography by using thebibliography environment as in
%
% \begin{thebibliography}{12}
% ...
% \bibitem{itemreference} D. E. Knudsen.
% {\em 1966 World Bnus Almanac.}
% {Permafrost Press, Novosibirsk.}
% ...
% \end{thebibliography}

% Here's where you specify the bibliography style file.
% The full file name for the bibliography style file 
% used for an ASME paper is asmems4.bst.
\bibliographystyle{References}

% Here's where you specify the bibliography database file.
% The full file name of the bibliography database for this
% article is asme2e.bib. The name for your database is up
% to you.
\bibliography{References}

%%%%%%%%%%%%%%%%%%%%%%%%%%%%%%%%%%%%%%%%%%%%%%%%%%%%%%%%%%%%%%%%%%%%%%
\appendix       %%% starting appendix
\section*{Appendix A: Head of First Appendix}
Avoid Appendices if possible.

%%%%%%%%%%%%%%%%%%%%%%%%%%%%%%%%%%%%%%%%%%%%%%%%%%%%%%%%%%%%%%%%%%%%%%


\end{document}
